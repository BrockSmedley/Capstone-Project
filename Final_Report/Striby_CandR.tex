\subsubsection{What technical information did you learn?}

\paragraph{Hardware}

The amount of information surrounding the hardware involved with this project was very 
interesting. Research on the NVIDIA Jetson TX2 was extremely eye-opening due to its 
endless solutions to countless applications. Then there's the specifications of the 
Jetson TX2 and how it has evolved from its predecessors. Information found while looking 
for an application specific carrier board was very intriguing, because of all the 
peripherals that accompanied each option. With each option was room for growth for 
future development to enhance the product. Then there was the CSI-based input for our 
captured images, something that most have in their pockets for the majority of the day. 
Its advantage was quickly obvious due to how fast it was capable of providing the data 
to the rest of the product's system. Even though I learned a lot, there is still much 
more to figure out regarding these topics and technologies.  \\

\paragraph{Software}

Learning the software required to produce a successful product never ended over the 
course of the academic school year. Upon learning about specific requirements for the 
project there was a quick immersion into figuring out how everything should come together. 
It all started with researching the JetPack L4T software development kit which has a 
treasure trove of tools such as programming libraries, multimedia API, and developer 
tools with limitless applications. Once we determined that our solution would use 
GStreamer for initial processing of the streaming media and OpenCV for image processing, 
there was a lot of investigating and research on how we would be able to bring these 
two pieces together. Our project was only capable of figuring out a simple solution 
to such a complex problem, and the sky is the limit in regard to composing more 
successful answers.  \\

\subsubsection{What non-technical information did you learn?}
%\paragraph{if necessary}

Throughout the entire academic school year, I was continuously shown how much 
documentation is necessary to start, support, and conclude a well thought and 
organized project. Initial reports helped ensure that our understanding of what 
the project's problem was made it easier to communicate with our client about 
the desired product. Going over the software design requirements helped provide 
a good breakdown of the various phases and tasks, and this helped make the 
project less overwhelming. Then the progress and final reports provided reflection on
the work performed and thoughtful forward thinking for future solutions to the 
product.  \\

Then the supporting documentation for easy reference ensured that every detail 
was accounted for and could be looked back upon. Our client and teaching assistant 
meetings introduced the importance 
of making note of everything discussed, and these notes were referred to during the 
entire project. Daily notes of tasking performed and lab notes of work contributed to 
the project made it easy to reference and replicate anything and everything. 
Due to the amount of information not everything could 
be written down, but with improved organization this will become a reality.  \\

\subsubsection{What have you learned about project work?}
%\paragraph{if necessary}

From the countless hours spent on this project came numerous lessons on what accompanies 
project work. The most important lesson learned was patience and that I need to 
develop more of it surrounding results and expectations. Not everything in this project 
could be understood at once and breaking it down even more would have helped contribute 
to creating more patience. Another very important lesson learned was remembering to 
take a step back to consider other options when a dead end may have been found. Finally, 
the last thing that I learned was that failure is expected sometimes, and as long as you 
learned why then it should not repeat itself.  \\

Overall, project work can be extremely discouraging, especially due to the amount of 
time spent on this project compared to the results produced. With many of the lessons 
previously discussed and others not mentioned, it can take many projects to become 
successful. But, it is also important to remember that the project can still be considered 
a success even though the desired results were not achieved. Some results are completely 
out of my control and it is important to remember this moving forward in my career. \\

\subsubsection{What have you learned about project management?}
%\paragraph{if necessary}

After enduring this project there are lots of room for improvement regarding my project 
management skills. Even though my organization skills are good, they were put to the test 
while working on the various aspects of the project. OneNote was a great tool to help 
document the project and without it this I would have been far less organized. 
Making daily and weekly goals based on the current project status would have prohibited 
more insite on current problems needing to be addressed. Holding members of the group 
accountable by assigning specific tasks could have prevented me from trying to balance 
a good majority of the work needing to be done. These skills can be taught in a 
classroom but there are so many circumstances that are unforeseen and it is very 
difficult to know how you will initially react to problems without experience handling 
them.   \\

\subsubsection{What have you learned about working in teams?}
%\paragraph{if necessary}

The group dynamic experienced while working on this project taught be a few new lessons 
and also showed me some areas that need improvement. Even though all group members have 
the same goal to achieve, we all have different expectations of ourselves. Figuring this 
out sooner and letting others understand my expectations may have helped promote others 
to contribute more to the group. Better communication skills all around would have been 
helpful over the life of this project. Everything from the medium being used to the 
soft skills while communicating. In my next project I plan to implement the request for 
read receipts from group members on important and emerging issues to promote more 
accountability.  \\

The most important lesson of all was the need for camaraderie, because due to its absense  
it felt as if the entire group was working on the project as an individual. 
With camaraderie the project would have been less of a chore and more enjoyable. 
Every accomplishment would have been shared and celebrated instead of quickly forgotten 
because of the next big task. Figuring this out earlier would have made the entire 
project much more memorable.  \\

\subsubsection{If you could do it all over, what would you do differently?}
%\paragraph{if necessary}

There are so many things that could have been done differently over the course of the 
project, and reflecting sooner may have resulted in a better project result. 
In the beginning after learning who my group members were, a better 
attempt in getting to know both of them  
would have helped figure out about their strengths and weaknesses. Our communications methods would 
involve ways to better track things needing to be done so that we each could hold
each other accountable. Instead of just stating what needs to be done, handing out tasks 
would have provoked more action instead of just reactions to repeated requests. 
Finding out how each group member collaborates the best and finding a middle ground 
comfortable for all may have enticed thoughtful brainstorming. So much could have 
been improved upon during the project and this class was a great way to learn 
things that are not taught in a classroom in hopes that the mistakes made will not be 
repeated as a working professional.  \\


