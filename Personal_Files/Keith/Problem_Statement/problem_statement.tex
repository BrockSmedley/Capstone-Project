\documentclass[letterpaper,10pt,serif,draftclsnofoot,onecolumn,compsoc,titlepage]{IEEEtran}

\usepackage{graphicx}                                        
\usepackage{amssymb}                                         
\usepackage{amsmath}                                         
\usepackage{amsthm}                                          
\usepackage{cite}
\usepackage{alltt}                                           
\usepackage{float}
\usepackage{color}
\usepackage{url}

\usepackage{balance}
\usepackage[TABBOTCAP, tight]{subfigure}
\usepackage{enumitem}

\usepackage{geometry}
\geometry{margin=.75in}
\usepackage{hyperref}
%\usetikzlibrary{shapes, positioning, calc}
\usepackage{caption}
\usepackage{listings}
%\usepackage[utf8]{inputenc}
%pull in the necessary preamble matter for pygments output

%% The following metadata will show up in the PDF properties
\hypersetup{
   colorlinks = true,
   citecolor = black,
   linkcolor = black,
   urlcolor = black,
   pdfauthor = {W Keith Striby Jr},
   pdfkeywords = {CS461 "Senior Project" Problem Statement},
   pdftitle = {CS 461 Problem Statement},
   pdfsubject = {CS 461 Problem Statement},
   pdfpagemode = UseNone
}

\parindent = 0.0 in
\parskip = 0.1 in
\title{Problem Statement}
\author{W Keith Stirby Jr \\ 09 October 2017 \\ CS-461, Fall 2017}
\begin{document}
\maketitle
\begin{abstract}

Creating complex engineered systems with high cost-to-benefit ratios can be difficult to sell to customers that operate and maintain a multi-billion dollar industry. When passengers purchase their plane tickets, they’re most likely cause for a delay to their final destination is weather delays. However the airline industry is capable of overlooking this causation due to the passenger’s agreement upon the completion of their plane ticket. As passengers in any situation we can only plan our trip, but unfortunately not the weather…… (more)\\

\end{abstract}

\section{Problem}

Rockwell Collins currently offers a head-up display (HUD) for airlines, which is a transparent display for pilots to navigate safely during take off and when landing at airports. These HUDs play a vital role for pilots as it allows them to navigate through weather conditions that were once never considered because of its images and indications. The HUD’s transparent screen aligns with and provides images of geographic and man-made features that the pilot may not be capable of seeing on the ground during inclement conditions, and the images and indications on the HUD enhances the visibility for the pilot. The downside of the HUD’s system is the cost to commercial airlines, and their unwillingness to pay regardless of the benefit the HUD provides operationally. Rockwell Collins is needs a HUD in which its system is affordable for commercial airlines.  \\

\section{Proposed Solution}

A HUD that has a system composed of affordable hardware that reduces the total cost for commercial airlines. The major components providing an output for the HUD are a processor core, camera interface board, and cameras. To maintain a reduction in cost for these components cannot be custom engineered and produced in-house by Rockwell Collins, and therefore will be a pre-made, off the shelf, system on a chip and mass-produced products available to the public. Once these three major components are capable of being integrated and are communicating successfully, a fused image from two or more camera inputs will be our final major milestone to prove that a HUD system can be economically made. \\

\section{Performance Metrics}

The following is a list of metrics that must be met in order for the project to be deemed a success.\\

\subsection{System Research}

The processor core, camera interface board, and cameras must be capable of being integrated for the system to produce an output to a screen. Researching compatible components for system integration will be the first major step to producing an output to screen. These components must also meet size, weight, power, and cost (SWAP-C) requirements due to the application for the HUD. \\

\subsection{System Integration}
Once the system components are finalized, purchased, and in-hand system integration is required for the major components to communicate. 

\subsection{System Output with One Camera}

\subsection{System Output with Two or More Cameras}

\subsection{System Output with up to Six Cameras}


\end{document}
