\documentclass[letterpaper,10pt,serif,draftclsnofoot,onecolumn,compsoc,titlepage]{IEEEtran}

\usepackage{graphicx}                                        
\usepackage{amssymb}                                         
\usepackage{amsmath}                                         
\usepackage{amsthm}                                          
\usepackage{cite}
\usepackage{alltt}                                           
\usepackage{float}
\usepackage{color}
\usepackage[hyphens]{url}
\usepackage{pgfgantt}
\usepackage{rotating}
\usepackage{enumitem}
\usepackage{gensymb}
\usepackage[T1]{fontenc}

\usepackage{balance}
\usepackage[TABBOTCAP, tight]{subfigure}
\usepackage{enumitem}

\usepackage{geometry}
\geometry{margin=.75in}
\usepackage{hyperref}
\usepackage{breakurl}
%\usetikzlibrary{shapes, positioning, calc}
\usepackage{caption}
\usepackage{listings}
%\usepackage[utf8]{inputenc}
%pull in the necessary preamble matter for pygments output

%% The following metadata will show up in the PDF properties
\hypersetup{
   colorlinks = true,
   citecolor = black,
   linkcolor = black,
   urlcolor = black,
   breaklinks = true,
   pdfauthor = {W Keith Striby Jr},
   pdfkeywords = {CS461 "Senior Project" Technology Review},
   pdftitle = {CS461 Technology Review},
   pdfsubject = {CS461 Technology Review},
   pdfpagemode = UseNone
}

\parindent = 0.0 in
\parskip = 0.1 in
\title{Technology Review: Multi-Camera, SoM Based, Real-Time Video Processing for UAS and VR/AR Applications}
\author{Area 51: W Keith Stirby Jr \\ 21 November 2017 \\ CS461, Senior Software Engineering Project, Fall 2017}
\begin{document}
\begin{titlepage}
\maketitle
\begin{abstract}

In this report are three technologies that will be necessary for our Capstone Project. 
Each technology was narrowed down to three options that produce similar 
results. Criteria is established, each option is explored and compared, and then 
a decision explaining why that particular option was selected. \\

\thispagestyle{empty}
\end{abstract}
\end{titlepage}
\newpage
\tableofcontents
\newpage

\section{Introduction}

The client for our project has requested that we design software for multi-camera video 
streaming, which will be used for UAS (Unmanned Aerial Systems) and VR/AR applications. 
The video output is expected to operate near real-time following image processing, and the multi-camera inputs are 
expected to be stitched and fused on the output display. There are SWaP-C (Size, Weight, Power,
and Cost) constraints and edge computation requirements due to our project's applications. \\

For the duration of the project my role will be to:
\begin{itemize}
	\itemsep-2em
	\item Fulfill the role as liaison between our group and the client. \\
	\item Research and procure hardware required for the product. \\
	\item Assist in software development for stitching and fusing video output. \\
	\item Assist in software development for image processing. \\
\end{itemize}

The remainder of this document investigates three technological areas involved with 
our project. Each section gives a brief description of the technology, explains the 
criteria our choices needed to meet due to the product's application, and displays why 
we would consider each of them. A discussion then follows to compare and contrast each 
technology, and a conclusion then explains why we made our final selection.\\

\section{System-on-Module}

\subsection{Overview}

Selecting our SoM will be the foundation for our project's hardware 
and software choices because of its influence on computational performance, image 
processing, and future product development and expansion. A SoM is capable of offering 
specialized applications a dense computer system in a small package while
maintaining the nature of a microcontroller. SoMs are typically built on a single 
circuit board roughly the size of a credit card, have low power consumption, and 
need to be tethered to a carrier board for input and output peripherals.\\

\subsection{Criteria}

The project requires near real-time image processing to occur in a system that is 
standalone, and conforms to our client's requirements for SWaP-C due to its intended application. For the image processing to 
occur anywhere a UAS travels, the performance of the SoM 
needs to be capable of edge computing while it's up in the clouds.
Multiple camera inputs will be processed through our SoM and their images will be 
stitched, fused, analyzed, and have high-resolution video output. This requires our 
SoM to have a fast, efficient CPU (Central Processing Unit) complex, GPU 
(Graphics Processing Unit)-accelerated computing capable for multimedia streaming, and 
a sizable amount of RAM (Random Access Memory) to aid in computational performance 
and therefore minimize latency.\\

\subsection{Potential Choices}

\subsubsection{NVIDIA Jetson TK1}

The NVIDIA Jetson TK1 is a developer kit built around the Tegra K1 SoC (System 
on Chip (CPU, GPU, ISP (In-system programming) in a single chip)), NVIDIA's first 
mobile processor featuring the same capabilities as a 
modern desktop GPU\cite{JetsonTK1, TK1Wiki}. The Jetson TK1 has the Linux4Tegra (L4T)
OS (operating system) pre-installed, and similar features that you would find on a 
Raspberry Pi and a PC such as GPIO, USB, display interface, SATA, mini-PCIe, and a
fan\cite{RPiHDWR, TK1Wiki}. The Tegra TK1 is built with NVIDIA's Kepler core which 
provides three times the performance than previous NVIDIA GPUs which enables high 
performance computing (HPC) architecture. It also has the NVIDIA CUDA platform to 
create compute-intensive software applications\cite{KeplerArch}. NVIDIA's BSP (board 
support package) and software stack for the Jetson TK1 features Linux Kernel version 
3.10.40 and OpenGL, OpenCL, and CUDA media APIs (Application Program Interfaces)
\cite{TK1L4T}. \\

\subsubsection{NVIDIA Jetson TX1}

The NVIDIA Jetson TX1 is an embedded SoM capable of supporting GPU computing, computer 
vision, graphics, and deep learning making it ideal for embedded AI computing
on an edge device. This SoM has a CPU with four 64-bit A57 ARM cores, and a GPU with 
NVIDIA Maxwell architecture and 256 CUDA cores that provides over 1 TeraFLOPs (floating 
point operations per second) of performance\cite{TX1Wiki, JetsonGenius}. The entire module 
draws just 10W of power making it very energy efficient, and can encode and decode 4K 
video. With 4 GB of RAM and 16 GB of eMMC storage, this module can load software 
applications to on-board storage and process data in RAM\cite{LinuxDot}. The module has 
a 400-pin Samtec board-to-board socket that allows it to connect to NVIDIA's development
board to make use of miscellaneous I/O ports, or to connect to carrier boards that are 
smaller and designed for specific-use applications.\\

NVIDIA provides the Jetson Software Development Pack (JetPack) 3.1, which introduces 
L4T 28.1. Key features in the JetPack 3.1 are TensorRT 2.1, cuDNN 6.0, VisionWorks 1.6, 
CUDA 8.0, Multimedia API, and L4T has Kernel 4.4 for the Jetson TX1. JetPack Developer 
Kit is flashed by the JetPack installer to load the latest OS image, which will install 
developer tools in the host PC and Developer Kit. The host PC is required to have 
Ubuntu Linux x64 (v14.04) to run the JetPack installer\cite{JetPack, JetPackRel}.\\ 

\subsubsection{NVIDIA Jetson TX2}

The NVIDIA Jetson TX2 is the latest SoM by NVIDIA, making it their fastest computing edge
device with 8 GB of RAM and 32 GB of eMMC, twice the memory and storage on the TX1. The 
module has a GPU with NVIDIA Pascal architecture and two NVIDIA Denver2 CPUs plus four 
ARM Cortex-A57 CPUs\cite{TX2Wiki, JetsonFAQ}. The Jetson TX2 can be optimized for 
real-time processing on a UAS, and is 
still power-efficient and small\cite{JetsonGenius}. Additionally, the TX2 has 
Dual Operating Modes, MAX-Q for maximum efficiency and MAX-P for maximum performance, 
where the module can draw 7.5 watts or 15 watts, respectively\cite{TechnoByte}. The TX2 
module has the same 400-pin Samtec board-to-board socket for connection to NVIDIA's 
Developer Kit or carrier board. \\

NVIDIA's JetPack 3.1 is also compatible with the TX2, but features Linux kernel 
4.4\cite{TX2Wiki, JetPackRel}.\\

\subsection{Discussion}

\begin{tabular}{|l|p{5cm}|p{5cm}|p{5cm}|}
	\hline
	\textbf{} & \textbf{Jetson TK1} & \textbf{Jetson TX1} & \textbf{Jetson TX2}\\
	\hline
	\textbf{CPU} & NVIDIA 4-Plus-1 Quad Core ARM Cortex-A15 "r3" @ 2.3 GHz & ARM Cortex-A57 (quad-core) @ 1.73GHz & ARM Cortex-A57 (quad-core) @ 2GHz + NVIDIA Denver2 (dual-core) @ 2GHz\\
	\hline
	\textbf{GPU} & NVIDIA Kepler GK20a with 192 CUDA Cores & 256-core Maxwell @ 998 MHz & 256-core Pascal @ 1300 MHz\\
	\hline
	\textbf{Memory} & 2GB DDR3L 933MHz EMC x16 using 64-bit Width & 4GB 64-bit LPDDR4 @ 1600MHz | 25.6 GB/s & 8GB 128-bit LPDDR4 @ 1866MHz | 59.7 GB/s\\
	\hline
	\textbf{Storage} & 16Gb eMMC 4.51 & 16Gb eMMC 5.1 & 32Gb eMMC 5.1\\
	\hline
	\textbf{Video} & 1080p30 (2x) & 4K x 2K 30 Hz Encode (HEVC), 4K x 2K 60 Hz Decode (10-Bit Support) & 4K x 2K 60 Hz Encode (HEVC), 4K x 2K 60 Hz Decode (12-Bit Support)\\
	\hline
	\textbf{CSI} & 2 fast CSI-2 MIPI camera ports (1x4 + 1x1) & 12 lanes MIPI CSI-2 | 1.5 Gb/s per lane | 1400 megapixels/sec ISP & 12 lanes MIPI CSI-2 | 2.5 Gb/s per lane | 1400 megapixels/sec ISP\\
	\hline
	\textbf{Display} & 1 full-size port & 2x HDMI 2.0 / DP 1.2 / eDP 1.2 | 2x MIPI DSI & 2x HDMI 2.0 / DP 1.2 / eDP 1.2 | 2x MIPI DSI\\
	\hline
	\textbf{Wireless} & 802.11a/b/g/n/ac 2x2 867 Mbps Bluetooth 4.0 & 802.11a/b/g/n/ac 2x2 867 Mbps | Bluetooth 4.0 & 802.11a/b/g/n/ac 2x2 867 Mbps | Bluetooth 4.1\\
	\hline
	\textbf{USB} & USB 3.0 + micro-AB 2.0 & USB 3.0 + 2.0 & USB 3.0 + 2.0\\
	\hline
	\textbf{PCIe} & x1 Lane & Gen 2 | 1x4 + 1X1 & Gen 2 | 1x4 + 1X1 or 2x1 + 1x2\\
	\hline
	\textbf{Other} & UART, I2C, GPIOs & UART, SPI, I2C, I2S, GPIOs & CAN, UART, SPI, I2C, I2S, GPIOs\\
	\hline
	\textbf{Socket} & & 400-pin Samtec board-to-board connector, 50x87mm & 400-pin Samtec board-to-board connector, 50x87mm\\
	\hline
	\textbf{Thermals} & & -25\degree C to 80\degree C & -25\degree C to 80\degree C\\
	\hline
	\textbf{Power} & 15W & 10W & 7.5W\\
	\hline
	\textbf{Dimensions} & 5" x 5" board & 1.96" x 3.42" module & 1.96" x 3.42" module\\
	\hline
	\textbf{Weight} & 120 grams & 88 grams & 85 grams\\
	\hline
	\textbf{Cost} & \$192 each (Developer Kit) & \$299 at 1K units (module) & \$399 at 1K units (module)\\
	\hline
\end{tabular}
\newline
\newline
\newline
Note: Operating Lifetime is 5 years for Jetson Commercial Products.\\
Table References: \cite{TK1Wiki, TK1Power, TK1Rev, JetsonFAQ, TegraK1, TK12Comp, JetsonGenius, TX1PS, TX2DS}\\

NVIDIA's Jetson TK1 Development Kit (dev kit) was used in comparison to the Jetson TX1 
and Jetson TX2 modules because the NVIDIA Tegra K1 processor is not available on a 
module. The size of the Jetson TK1 dev kit puts it at a 
disadvantage when comparing it to the size, weight, and max power draw of the Jetson 
TX1 and TX2 modules. The major disadvantage that the TK1 has in comparison to the TX1 
and TX2 is the amount of CSI cameras it can accommodate. \\

NVIDIA was capable of doubling or enhancing all active processing components on the 
TX2 in comparison to the TX1, and both modules have the same dimensions. With NVIDIA's 
latest release of Jetpack 3.1 being compatible for both TX1 and TX2 both modules 
have near-equal software support, where most differences are application 
versions\cite{TX1Wiki, TX2Wiki, JetPackRel}. \\  

\subsection{Conclusion}

The final selection for our product's SoM is between the Jetson TX1 and TX2, and will 
be dependent on the software development and carrier board testing. Both the 
Jetson TX1 and TX2 offer great SWaP-C requirements for our product's application and 
are compatible with the carrier boards that have been selected to test. Additionally, 
both the Jetson TX1 and TX2 are capable of supporting up to six cameras through their 
12 CSI(Cameral Serial Interface)-2 lanes. \\

\section{Carrier Board}

\subsection{Overview}

Carrier boards (also known as baseboards) are designed with application-specific 
interfaces and peripherals, and offer scalability and flexibility due to the target 
application only being supported. A carrier board is always used in tandem with a SoM, 
and by separating the two a SoM can be upgraded without the possible redesign of the 
carrier board\cite{ToradexSBC, ArrowCB, ToradexCBQ}.\\

\subsection{Criteria}

The application of our product requires a carrier board that is capable of interfacing 
with multiple high-resolution cameras to produce a quality video output. The NVIDIA Jetson TX1 
and TX2 allow camera input through: USB 2.0 and 3.0 ports, MIPI CSI-2 camera ports, and 
potentially through the mini-PCIe ports\cite{JetsonCams}. MIPI CSI-2 cameras are roughly a few millimeters 
in size and allow for quick image processing due to the images being processed directly 
through the SoM's ISP. The MIPI CSI-2 camera interface is capable of supporting 1080p, 
4k, 8k video and it's the most widely used camera interface for mobile 
applications\cite{MIPIOverview}. With that, our carrier board must be capable of interfacing  
six CSI-2 camera inputs simultaneously. \\

\subsection{Potential Choices}

\subsubsection{Auvidea J106 Carrier Board}

The Auvidea J106 carrier board plugs into the 400-pin Samtec board-to-board connector 
on the Jetson TX1 and TX2, and is the same shape and size as both modules. 
It supports up to six MIPI CSI-2 15 pin camera input connections to our 
desired NVIDIA Jetpack SoM and also has an IMU (inertial measurement Unit). Additional 
interfaces featured on the J106 are: microSD, USB 
3.0, Micro USB, UART, 1 Gb Ethernet, I2C, SATA, 4x PCIe, CAN, and a mini 
HDMI\cite{AuvideaJ106}. The J106 board costs \$240 each at 1K units\cite{AuvideaQuote}.\\

The Auvidea M100 and M110 motherboards connects to the J106 board and all are the same 
size. Both motherboards are port extenders for the J106, which can allow the use of 
additional peripherals for input processing in the Jetson TX1 or 
TX2\cite{AuvideaMBoards}. \\

\subsubsection{Colorado Engineering, Inc X-Carrier Board}

The X-Carrier Board designed and by Colorado Engineering, Inc (CEI) is capable of 
supporting six 15 pin MIPI CSI-2 cameras through the Auvidea J20-2 camera adapter. The 
J20-2 board plugs into the bottom of the X-Carrier, and the Jetson TX1 or TX2 plugs 
into the 400-pin Samtec board-to-board connector on top. A Jetson TX1 or TX2 board does 
mount to the X-Carrier board but the X-Carrier is slightly larger. The X-Carrier has a built-in IMU 
and the following interfaces: HDMI 2.0, MicroSD, SATA 2.0, USB 2.0, USB 3.0, 1 Gb 
Ethernet, x1 and x4 PCIe Gen 2, 2x UART, 2x SPI, 3x I2C, 3x I2S, and GPIOs. CEI is  
developing various I/O modules to connect to the X-Carrier and include: radar, comms, 
active IR, passive IR, CANBus, and others\cite{CEIXpdf, J20TechRef, AuvideaJ20}. The X-Carrier costs \$285 each 
and the J20-2 costs \$129 each, both at 1K units\cite{SpacelyQuote, AuvideaQuote}. \\
	
\subsubsection{Connect Tech, Inc Spacely Carrier}

The Spacely Carrier designed by Connect Tech, Inc has six built-in MIPI CSI-2 camera 
inputs with 30 pins and offers a built-in expansion for a GPS/GNSS module. The NVIDAIA 
Jetson TX1 or TX2 module plugs into the 400-pin Samtec board-to-board connector on top 
of the Spacely Carrier. The following interfaces are on the Spacely Carrier: HDMI, 2x 
Gigabit Ethernet, 2x Micro USB 3.0, 2x USB 2.0, 1x USB OTG, 1x USB 2.0 to Mini-PCIe 
Slot, 1x mSATA, 2x 3.3V from Jetson UART0 and UART1, 1x Mini-PCIe, 1x microSD, 1x CAN 2.b 
Port, 1x I2C Link, 1x SPI Channel, and 14x GPIO\cite{SpacelyUG}. The Spacely Carrier 
costs \$411 each at 1K units.\\

\subsection{Discussion}

\begin{tabular}{|l|p{5cm}|p{5cm}|p{5cm}|}
	\hline
	\textbf{} & \textbf{J106 Carrier} & \textbf{X-Carrier} & \textbf{Spacely Carrier}\\
	\hline
	\textbf{CSI} & 6 2-lane CSI-2 (15 pin) & 6 2-lane CSI & 6 x2 Lane MIPI CSI-2 or 3 x4 Lane MIPI CSI-2 (30 pin)\\
	\hline
	\textbf{USB} & 2x USB 3.0, 1x Micro USB & 1x USB 3.0, 1x Mini-USB 2.0 & 2x Micro USB 3.0, 2x USB 2.0, 1x USB OTG, 1x USB 2.0 to Mini-PCIe Slot \\
	\hline
	\textbf{SATA} & 1x SATA & 1x SATA 2.0 & 1x mSATA Full Size \\
	\hline
	\textbf{Display} & 1x mini HDMI & 1x HDMI 2.0 & 1x HDMI \\
	\hline
	\textbf{PCIe} & 4x PCIe & x1 and x4 PCIe Gen 2 & 1x Mini-PCIe \\
	\hline
	\textbf{Other} & UART, Gb Ethernet, I2C, 1x CAN & MicroSD, 1 Gb Ethernet, 2x UART, 2x SPI, 3x I2C, 3x I2S, and GPIOs & 2x 3.3V from Jetson UART0 and UART1, 1x MicroSD, 1x CAN 2.b Port, 1x I2C Link, 1x SPI Channel, and 14x GPIO \\
	\hline
	\textbf{Thermals} &  & -20\degree C to 55\degree C & -40\degree C to 85\degree C \\
	\hline
	\textbf{Voltage} & 12V (4 pin), range: 7V to 17V & 12V to 19V & 12V to 22V \\
	\hline
	\textbf{Dimensions} & 1.96" x 3.42" board & 3.43" x 2.8" board & 4.92" x 3.74" board \\
	\hline
	\textbf{Weight} & 2.6 grams & 6.5 grams & 90.7 grams \\
	\hline
	\textbf{Cost} & \$240 each at 1K units & \$285 each at 1K units & \$411 each at 1K units \\
	\hline
\end{tabular}	
\newline
\newline
\newline
Note: The Auvidea M100 and M110 motherboards are \$129 each at 1K units, and the 
J20-2 board is \$129 each at 1K units. \\
Table References: \cite{AuvideaJ106, MouserJ106, CEIX, CEIXpdf, SpacelyUG, SpacelyQuote, CEIQuote, AuvideaQuote}\\

The J106 carrier board size is form fitting with the Jetson TX1 or TX2 modules and 
maintains this size when even used with the Auvidea M100 or M110 motherboards, unlike 
the X-Carrier's and Spacely Carrier's larger board dimensions. However, the J106 carrier board 
offers limited interfaces in comparison to the X-Carrier and Spacely Carrier boards due 
to its size and relies on its 4x PCIe slots for future options like a GPS/GNSS module. The 
X-Carrier does rely on the Auvidea J20-2 board to connect to for supporting  
camera input whereas the J106 and Spacely Carrier have all six MIPI CSI-2 ports directly 
installed. The major benefit to the Spacely Carrier's size is that it has a large 
amount of USB ports and miscellaneous I/O ports and a built-in expansion for a 
GPS/GNSS module. Although, the J106 and X-Carrier Boards have built-in IMUs whereas the 
Spacely Carrier does not.\\

\subsection{Conclusion}

The final selection for our product's carrier board is dependent on multiple variables, 
all of which will be determined during software development and testing. All three of 
the companies 
designed and developed their carrier boards for specific use with the Jetson TX1 and 
TX2 modules. Each carrier board offers subtle and unique 
advantages over the other, and it's difficult to say which one will be capable of 
providing the best solution without adequate testing. \\

\section{Camera Case}

\subsection{Overview}

The Raspberry Pi (v2) cameras selected for developing our software have an exposed circuit 
board and camera lens. These cameras should be enclosed for protection and mounted to 
provide consistent testing of the software being produced. Camera mounting and positioning
will need to remain in the same known setup while developing software for multiple camera 
input. \\

\subsection{Criteria}

The case needs to be capable of performing two basic functions for the camera, which 
are protecting the camera and holding the camera securely. The case should be specifically 
designed for our Raspberry Pi cameras to ensure these two basic functions are easily 
met. For mounting of the case to another object it will be beneficial for it to 
have screw holes on it. \\

\subsection{Potential Choices}

\subsubsection{Adjustable Pi Camera Mount}

This camera mount made by Pimoroni is as basic as they come and are \$5 each. The mount 
consists of a two-piece bracket, one to mount the camera to and another to fix the 
mount. However, the support bracket and camera bracket snap together. This camera 
mount includes screws and hex-nuts to secure the camera to the bracket\cite{AdjCamMt}. \\

\subsubsection{Raspberry Pi Camera Case/Enclosure}

SB Components created a camera case specifically designed for the Raspberry Pi camera 
and snaps together to firmly hold the camera in place. The case has two screw holes for 
mounting, comes with screws and tape, and are \$6 each\cite{BlueCase}. \\

\subsubsection{Adafruit Raspberry Pi Camera Board Case}

This Adafruit Raspberry Pi enclosure is sleek, snaps closed, and has four pegs inside making a 
secure hold on the camera. To mount the case are two slots and a 
one-quarter inch tripod mount nut. A hole on the lid of the case to prevents 
distorting camera images. Our Raspberry Pi Camera v2 is slightly thicker than the 
original camera that the case was designed for, and it is recommended to remove a spacer from 
the circuit board to prevent blurry images. These cases are priced at \$3 each\cite{adafruitCase}.\\

\subsection{Discussion}

Each option provides a different stance on simplicity, caters to custom mounting 
options, and is specifically designed for Raspberry Pi. The adjustable mount 
by Pimoroni offers no protection unlike SB Components and Adafruit's cases. The 
mounting holes on SB Components enclosure does come with screws for them but with 
just a little extra effort the Adafruit case can be just as secure. Adafruit's 
case has a sleek and minimalist design and may be more versatile for mounting, but 
the big blocky SB Components case positions the camera vertically and has the 
screw holes on its horizontal surface. \\

\subsection{Conclusion}

The SB Components camera case was chosen because its enclosure can hold the camera's 
field of view vertical while the mounting holes are on the horizontal axis. Pimoroni's
mount was ruled out since the two camera cases have the best mounting options for 
securing the cameras in place. The Adafruit's case will require 
more effort and creativity to mount the cameras securely for testing stitched or 
fused camera inputs, and is why SB Components case is slightly more appealing.\\

\nocite{*}
\newpage
\bibliographystyle{ieeetr}
\bibliography{Technology_Review_Striby}
\end{document}