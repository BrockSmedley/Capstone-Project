\documentclass[onecolumn, draftclsnofoot,10pt, compsoc]{IEEEtran}
\usepackage{graphicx}
\usepackage{url}
\usepackage{setspace}

\usepackage{geometry}
\geometry{textheight=9.5in, textwidth=7in}

% 1. Fill in these details
\def \CapstoneTeamName{		Area 51}
\def \CapstoneTeamNumber{		51}
\def \GroupMemberOne{			Shu-Ping Chien}
\def \GroupMemberTwo{			Brock Smedley}
\def \GroupMemberThree{			William Striby Jr}
\def \CapstoneProjectName{		Multi-camera, System-on-Chip (SoC) based, real-time video processing for UAS and VR/AR applications}
\def \CapstoneSponsorCompany{	Rockwell Collins HGS}
\def \CapstoneSponsorPerson{		Carlo Tiana}

% 2. Uncomment the appropriate line below so that the document type works
\def \DocType{		Problem Statement
				%Requirements Document
				%Technology Review
				%Design Document
				%Progress Report
				}

\newcommand{\NameSigPair}[1]{\par
\makebox[2.75in][r]{#1} \hfil 	\makebox[3.25in]{\makebox[2.25in]{\hrulefill} \hfill		\makebox[.75in]{\hrulefill}}
\par\vspace{-12pt} \textit{\tiny\noindent
\makebox[2.75in]{} \hfil		\makebox[3.25in]{\makebox[2.25in][r]{Signature} \hfill	\makebox[.75in][r]{Date}}}}
% 3. If the document is not to be signed, uncomment the RENEWcommand below
\renewcommand{\NameSigPair}[1]{#1}

%%%%%%%%%%%%%%%%%%%%%%%%%%%%%%%%%%%%%%%
\begin{document}
\begin{titlepage}
    \pagenumbering{gobble}
    \begin{singlespace}
    	%\includegraphics[height=4cm]{coe_v_spot1}
        \hfill
        % 4. If you have a logo, use this includegraphics command to put it on the coversheet.
        %\includegraphics[height=4cm]{CompanyLogo}
        \par\vspace{.2in}
        \centering
        \scshape{
            \huge CS Capstone \DocType \par
            {\large\today}\par
            \vspace{.5in}
            \textbf{\Huge\CapstoneProjectName}\par
            \vfill
            {\large Prepared for}\par
            \Huge \CapstoneSponsorCompany\par
            \vspace{5pt}
            {\Large\NameSigPair{\CapstoneSponsorPerson}\par}
            {\large Prepared by }\par
            Group\CapstoneTeamNumber\par
            % 5. comment out the line below this one if you do not wish to name your team
            \CapstoneTeamName\par
            \vspace{5pt}
            {\Large
                \NameSigPair{\GroupMemberOne}\par
                \NameSigPair{\GroupMemberTwo}\par
                \NameSigPair{\GroupMemberThree}\par
            }
            \vspace{20pt}
        }
        \begin{abstract}
        % 6. Fill in your abstract
        	This project will figure out a way to replace the transparent display devices which is used as enhanced vision system for airlines with a stand-alone TX2 to communicate with an interface board smaller than DevKit to adjust the large size and high cost problem. The goal is to choose proper hardware basis with at least two cameras to display images to single stream and reach all real-time processing requirements from client, which asks for time to research in hardware devices and software computation to optimize the system. Once the baseline goal is achieved, there are also stretch goals to improve the device with up to six cameras.
        \end{abstract}
    \end{singlespace}
\end{titlepage}
\newpage
\pagenumbering{arabic}
%\tableofcontents
% 7. uncomment this (if applicable). Consider adding a page break.
%\listoffigures
%\listoftables
\clearpage

% 8. now you write!
\section{The First Section}
The transparent display devices from Rockwell Collins company are used for pilots to assist analyzing environment during flight, so pilots can read the data of environment such as height, speed, and direction printed on the transparent screen with view. Moreover, the device with enhanced vision system is able to provide clear and detailed pictures even in bad weather because sensors record longwave, shortwave, and visible light. For example, people can not have clear vision when flight in foggy weather, but the demonstrated view can be generate from database with the enhanced vision system. By detection of multispectral data, the system compute and form the human readable view in all weather condition.

However, the cost of the device is too expensive to be produced for public airlines, and an idea to cut down the cost is to build multispectral camera systems based on the same features of transparent display device. This project will figure out a way to create self system which allows real-time processing analysis on perceptual sense, software engineering sense, and in signal processing sense. The desired outcomes should be a stand-alone TX2, which is a development platform for visual computing module, to communicate with an interface board smaller than DevKit. The stand-alone device is asked to be designed as an individual devices that data stream can be analysed without connecting to computer. Since the NVIDIA Jetson TX2 supports up to six cameras, it can build a video processor and display/recording system suitable as a UV/light air vehicle payload. This project also requires the device to provide attractive size, weight, power and cost (SWaP-C) for this application.

To solve the problem, the client sets smaller steps and basic requirements to achieve. Since the TX2 module is not the only choice of the multi-cameras, the team members should start with researching on details of hardware to ensure whether the specification of product matched the requirements. The current requirements should be achievable with size inside the interface, power under x watts, and 30 Hz of runtime selectable video output. One of the camera the client uses is FLIR Boson, a camera with low SWaP (starting at 7.5 grams and 4.9 cubic centimeters) and specifications fits the requirements. The team would update the weekly improvement with assigned research for each individual and communicate with the client monthly to trace the process. After everything in the first step matches the requirements and is recognized by the client, the team moves to software designing step to display images into single stream.

Then in the step of displaying images, it requires for image quality and real-time processing. First, in order to acquire readable images, the team is asked to learn knowledge about graphing skills such as transferring images into bits of data, reading and analyzing each pixel specification of sensors, and fusing images together. In the basic outline, the basic cameras can be break down with at least one infrared and visible band for each. For real-time processing, it needs to determine the amount of time required by the algorithm to complete all required transferring and processing of image data, and ensure it is less than the allotted time for processing. For example, it can be compared with the NVIDIA Jetson TX1 CPU, either the maximum transfer rate 800 Mbps with wifi or 24 Mbps with bluetooth provides option to transfer data stream.

The benefit of building the multi-camera, system-on-Chip based, real-time video processing helps to cut down the cost instead of transparent display device. Therefore, safety of airlines has been enhanced if the devices become available for every airline because precise data can be presented for pilots even in bad weather condition. On the other hand, since the device is expected to be designed with the life cycle in two years, it means that each chip costs lower and the whole device can be improved frequently. With the self development system, it is easier to follow the developing steps to modify shortages.

In order to achieve performance metrics, the client provides baseline goals and stretch goals, and the baseline goals must be achieved. Since there are two kinds of band, infrared and visible, from images, the baseline goals requires at least two cameras on the self contained device which matches all requirements. Then the stretch goals are able to be improved up to six cameras, so the images can be generated in more precise way.

Combination of multiple cameras on stand alone interface board gives the possibility for airlines to acquire visual information in proper size with realistic cost. In this proposal, the multispectral camera systems developed on NVIDIA Jetson TX2 module is analyzed for the feasibility and steps for implementation are proposed.
\end{document}
