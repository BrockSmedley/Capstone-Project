\documentclass[letterpaper,10pt,serif,draftclsnofoot,onecolumn,compsoc,titlepage]{IEEEtran}

\usepackage{graphicx}                                        
\usepackage{amssymb}                                         
\usepackage{amsmath}                                         
\usepackage{amsthm}                                          
\usepackage{cite}
\usepackage{alltt}                                           
\usepackage{float}
\usepackage{color}
\usepackage[hyphens]{url}
\usepackage{pgfgantt}
\usepackage{rotating}
\usepackage{enumitem}
\usepackage{gensymb}
\usepackage[T1]{fontenc}

\usepackage{balance}
\usepackage[TABBOTCAP, tight]{subfigure}
\usepackage{enumitem}

\usepackage{geometry}
\geometry{margin=.75in}
\usepackage{hyperref}
\usepackage{breakurl}
%\usetikzlibrary{shapes, positioning, calc}
\usepackage{caption}
\usepackage{listings}
%\usepackage[utf8]{inputenc}
%pull in the necessary preamble matter for pygments output

%% The following metadata will show up in the PDF properties
\hypersetup{
   colorlinks = true,
   citecolor = black,
   linkcolor = black,
   urlcolor = black,
   breaklinks = true,
   pdfauthor = {Shu-Ping Chien, Brock Smedley, W Keith Striby Jr},
   pdfkeywords = {CS461 "Senior Project" Progress Report},
   pdftitle = {CS461 Progress Report},
   pdfsubject = {CS461 Progress Report},
   pdfpagemode = UseNone
}

\parindent = 0.0 in
\parskip = 0.1 in
\title{Progress Report: Multi-Camera, SoM Based, Real-Time Video Processing for UAS and VR/AR Applications}
\author{Area 51: Shu-Ping Chien, Brock Smedley, W Keith Stirby Jr \\ 02 December 2017 \\ CS461, Senior Software Engineering Project, Fall 2017}


\begin{document}
\begin{titlepage}
\maketitle

\begin{abstract}


\thispagestyle{empty}
\end{abstract}
\end{titlepage}

\newpage
\tableofcontents

\newpage

\section{Introduction}

\subsection{Purpose}

Our project is required to provide a video output at near real-time from a 
multi-camera input utilizing an NVIDIA Jetson TX1 or TX2 system-on-module (SoM). 
The software produced will perform image processing and edge computing on the 
camera input to display a visually enhanced and stitched video output. 
Size, weight, power, and cost (SWaP-C) requirements for the project are due 
to its application being for UAS and VR/AR, and from our system utilizing a 
mass-produced SoM, particularly the Jetson TX1 or TX2.  

\subsection{Scope}
 
SHU-Ping or BROCK - PLEASE WRITE
Software scope/requirements here. (Take pixel streams, do the 
hokey pokey, and we have low latency, wicked fast video at near real-time speed.)
 
Application specific hardware and software are vital for the project due the the 
system requirements and constraints of the project.
The NVIDIA Jetson TX1 and TX2 run on an Ubuntu based Linux for Tegra (L4T) operating 
system created specifically for producing customized imaging, installed on a 
GPU-accelerated dual-core CPU with dual Image Signal Processors (ISPs). 
NVIDIA provides the Jetson Software Development Pack (JetPack) 3.1 which has L4T 28.1 
and is capable of supporting a multitude of multimedia and image processing API. 
The cameras will utilize the most widely used camera interface for mobile applications, 
MIPI CSI-2, which is capable of supporting 1080p, 4k, and 8k video. 
A carrier board will connect the cameras to the Jetson TX1 or TX2 that also provides 
additional application-specific interfaces and peripherals. 


\subsection{Overview}



\section{Week 1}

\subsection{Summary}

The first week of the term project choices were presented and preferences were
submitted. No project specific progress was made. 

\subsection{Retrospective}

\begin{tabular}{|p{0.3\linewidth}|p{0.3\linewidth}|p{0.3\linewidth}|}
   \hline
   \textbf{Positives} & \textbf{Deltas} & \textbf{Actions}\\ 
   \hline
   N/A 
   & N/A 
   & N/A \\
   \hline
\end{tabular}

\section{Week 2}

\subsection{Summary}

The group was assigned the project specified and immediately began coordinating 
with the project's client, Carlo Tiana of Rockwell Collins. 
An email was written to Tiana to make first contact, exchange contact 
information,
established preferred contact medium, and began inquiring about our first meeting.
Introductions began with the group for each other to become acquainted with 
each other. 
The group shared their schedules on Google Calendar with each other and Tiana 
to coordinate weekly meetings for the term. 
Preliminary research on the NVIDIA Jetson TX1 and TX2 module, real-time video 
processing, and processing multispectral camera systems was performed by group members 
to prepare for the meeting scheduled with Tiana on Monday of Week 3. \\


\subsection{Retrospective}

\begin{tabular}{|p{0.3\linewidth}|p{0.3\linewidth}|p{0.3\linewidth}|}
   \hline
   \textbf{Positives} & \textbf{Deltas} & \textbf{Actions}\\ 
   \hline
   The group was assigned to the project and members were introduced to one another. 
   Contact was initiated with the project's client and a meeting was scheduled. 
   Research began on the topics surrounding the project's title to prepare 
   for our initial meeting occurring next week. 
   & 
   N/A 
   & 
   N/A \\
   \hline
\end{tabular}


\section{Week 3}

\subsection{Summary}

An initial meeting was held with the client Tiana a principle engineer and expert 
in aviation vision systems, and Weston Lahr a software engineer 
for Rockwell Collins on Monday. Initial introductions occurred between the group, 
Tiana, and Lahr. 
Tiana presented to the group background information on the motivation for the project, 
provided a more elaborate description of what is expected of the group's product, and 
established objectives for the course of the project. The group inquired about 
information required to be answered in the Problem Statement document during the meeting 
as well. \\

Following the meeting the group began researching project details explained by Tiana 
and Lahr. Due to concerns regarding hardware acquisition expressed by Tiana, the group 
began researching system components required to satisfy the project's objectives. 
Project details surrounding the product's preferred SoM being the Jetson TX1 or TX2, 
the group researched the module and potential carrier boards that support a MIPI 
CSI-2 specification. \\

The group created a dedicated Google Drive directory to share project documents and 
research with Tiana and Lahr. A GitHub repository was created and shared to 
collaborate required documents among group members. The Problem Statement draft was 
written. The group had an initial meeting with their TA, Daniel Lin, and meetings 
were scheduled to occur every Friday afternoon for the Fall Term.\\

\subsection{Retrospective}

\begin{tabular}{|p{0.3\linewidth}|p{0.3\linewidth}|p{0.3\linewidth}|}
   \hline
   \textbf{Positives} & \textbf{Deltas} & \textbf{Actions}\\ 
   \hline
   An initial meeting was held with the client Tiana, and Lahr who is a software 
   engineer with Rockwell Collins. Project details, motivations, and expectations 
   were established, and research began for the group based on the information learned. 
   Hardware research was initiated based on project requirements, specifications, and 
   contraints. The main hardware components consists of the SoM, a carrier board with 
   the MIPI CSI-2 interface, and CSI-2 cameras. 
   The group began drafting the Problem Statement and emailed questions 
   to Tiana to obtain answers needed to write the document. 
   & 
   N/A
   & 
   N/A \\
   \hline
\end{tabular}

\subsection{Summary}

\section{Week 4}

\subsection{Summary}

Research continued on topics surrounding the project which includes: the Jetson TX1 and 
TX2, carrier boards with a MIPI CSI-2 interface compatible with the TX1 and TX2 
capable of supporting multiple cameras, image processing at real-time, software support 
for the TX1 and TX2, the MIPI CSI-2 standard, and cameras with a MIPI CSI-2 interface. \\

The group continued to write their Problem Statement which involved: researching 
examples from previous Capstone Projects, obtaining feedback from the writing center 
on campus, and emailing questions to Tiana. 
The Problem Statement was submitted for review to the professors and client once 
complete. \\

The NVIDIA Jetson TX2 module and development kit was obtained from Kevin McGrath for 
the group to begin setting up the module.
A meeting was scheduled with Tiana and the group for Tuesday of Week 5 and the weekly 
meeting with Lin occurred on Friday.\\

\subsection{Retrospective}

\begin{tabular}{|p{0.3\linewidth}|p{0.3\linewidth}|p{0.3\linewidth}|}
   \hline
   \textbf{Positives} & \textbf{Deltas} & \textbf{Actions}\\ 
   \hline
   anything good that happened 
   & 
   changes that need to be implemented 
   & 
   specific actions that will be implemented in order to create the necessary 
   changes \\
   \hline
\end{tabular}

\section{Week 5}

\subsection{Summary}

The group's second meeting was held with Tiana on Tuesday of Week 5. The meeting was 
virtual and the topics were: feedback from Tiana regarding the Problem Statement, 
hardware for the project was discussed and Tiana presented some carrier board information 
to assist our research, and the group asked about information necessary to complete 
the Requirements Document. \\

Based off Tiana's information discussed during the meeting regarding carrier boards,
more research was performed. Potential carrier boards were pursued after and three 
were selected to be ordered. Additional hardware was selected to support those three 
carrier board selection and includes cameras, cables, port converters, and 
motherboards. Coordination occurred between the group and McGrath to order all 
hardware. \\

The group wrote and submitted their rough draft for the Requirements Document. The 
document was written based on information from meetings with Tiana, email 
communicaitons between the group and Tiana, and hardware research.  The group had 
their weekly Friday meeting with Lin. \\

\subsection{Retrospective}

\begin{tabular}{|p{0.3\linewidth}|p{0.3\linewidth}|p{0.3\linewidth}|}
   \hline
   \textbf{Positives} & \textbf{Deltas} & \textbf{Actions}\\ 
   \hline
   anything good that happened 
   & 
   changes that need to be implemented 
   & 
   specific actions that will be implemented in order to create the necessary 
   changes \\
   \hline
\end{tabular}

\section{Week 6}

\subsection{Summary}

Completing the Requirements Document was the group's primary concern due to the 
final draft needing to be submitted on Friday of Week 6. Requested feedback 
from Tiana and McGrath before submitting the final draft. The group continued to 
work on the Requirements Document through the weekend. \\

It was confirmed by McGrath that hardware for the project had been ordered. The group 
had their weekly Friday meeting with Lin. \\

\subsection{Retrospective}

\begin{tabular}{|p{0.3\linewidth}|p{0.3\linewidth}|p{0.3\linewidth}|}
   \hline
   \textbf{Positives} & \textbf{Deltas} & \textbf{Actions}\\ 
   \hline
   anything good that happened 
   & 
   changes that need to be implemented 
   & 
   specific actions that will be implemented in order to create the necessary 
   changes \\
   \hline
\end{tabular}

\section{Week 7}

\subsection{Summary}

The Requirements Document was submitted following feedback from Tiana and McGrath, 
and work immediately began on the Technology Review. The group collaborated on 
technologies involved in the project and divided up three technologies for each 
group member to research and report on. The weekly meeting with Lin did not happen 
due to Veteran's Day Holiday. Instead the group communicated with Lin via email. \\

\subsection{Retrospective}

\begin{tabular}{|p{0.3\linewidth}|p{0.3\linewidth}|p{0.3\linewidth}|}
   \hline
   \textbf{Positives} & \textbf{Deltas} & \textbf{Actions}\\ 
   \hline
   anything good that happened 
   & 
   changes that need to be implemented 
   & 
   specific actions that will be implemented in order to create the necessary 
   changes \\
   \hline
\end{tabular}

\section{Week 8}

\subsection{Summary}

The group continued to work on the Technology Review, and conversed about each group 
member's role in the project based on research areas for the review. Chien's research 
is on media streaming, image processing, and data storage. Smedley's research is on 
development technologies in the Jetson TX1 and TX2, system interfaces, and operating 
systems. Striby's research is on SoMs, carrier boards, and camera cases for mounting. \\

Hardware arrived and the group coordinated with McGrath to obtain everything. Carlo was 
informed of everything ordered thus far and what the group had possession of. Smedley 
retrieved the Jetson TX2 SoM from Striby and began configuring the device. The group 
had their weekly Friday meeting with Lin.\\

\subsection{Retrospective}

\begin{tabular}{|p{0.3\linewidth}|p{0.3\linewidth}|p{0.3\linewidth}|}
   \hline
   \textbf{Positives} & \textbf{Deltas} & \textbf{Actions}\\ 
   \hline
   anything good that happened 
   & 
   changes that need to be implemented 
   & 
   specific actions that will be implemented in order to create the necessary 
   changes \\
   \hline
\end{tabular}

\section{Week 9}

\subsection{Summary}



\subsection{Retrospective}

\begin{tabular}{|p{0.3\linewidth}|p{0.3\linewidth}|p{0.3\linewidth}|}
   \hline
   \textbf{Positives} & \textbf{Deltas} & \textbf{Actions}\\ 
   \hline
   anything good that happened 
   & 
   changes that need to be implemented 
   & 
   specific actions that will be implemented in order to create the necessary 
   changes \\
   \hline
\end{tabular}

\section{Week 10}

\subsection{Summary}



\subsection{Retrospective}

\begin{tabular}{|p{0.3\linewidth}|p{0.3\linewidth}|p{0.3\linewidth}|}
   \hline
   \textbf{Positives} & \textbf{Deltas} & \textbf{Actions}\\ 
   \hline
   anything good that happened 
   & 
   changes that need to be implemented 
   & 
   specific actions that will be implemented in order to create the necessary 
   changes \\
   \hline
\end{tabular}


\nocite{*}

\end{document}