\documentclass[letterpaper,10pt,serif,draftclsnofoot,onecolumn,compsoc,titlepage]{IEEEtran}

\usepackage{graphicx}                                        
\usepackage{amssymb}                                         
\usepackage{amsmath}                                         
\usepackage{amsthm}                                          
\usepackage{cite}
\usepackage{alltt}                                           
\usepackage{float}
\usepackage{color}
\usepackage[hyphens]{url}
\usepackage{pgfgantt}
\usepackage{rotating}
\usepackage{enumitem}
\usepackage{gensymb}
\usepackage[T1]{fontenc}

\usepackage{balance}
\usepackage[TABBOTCAP, tight]{subfigure}
\usepackage{enumitem}

\usepackage{geometry}
\geometry{margin=.75in}
\usepackage{hyperref}
\usepackage{breakurl}
%\usetikzlibrary{shapes, positioning, calc}
\usepackage{caption}
\usepackage{listings}
%\usepackage[utf8]{inputenc}
%pull in the necessary preamble matter for pygments output

%% The following metadata will show up in the PDF properties
\hypersetup{
   colorlinks = true,
   citecolor = black,
   linkcolor = black,
   urlcolor = black,
   breaklinks = true,
   pdfauthor = {Shu-Ping Chien, Brock Smedley, W Keith Striby Jr},
   pdfkeywords = {CS461 "Senior Project" Progress Report},
   pdftitle = {CS461 Progress Report},
   pdfsubject = {CS461 Progress Report},
   pdfpagemode = UseNone
}

\parindent = 0.0 in
\parskip = 0.1 in
\title{Progress Report: Multi-Camera, SoM Based, Real-Time Video Processing for UAS and VR/AR Applications}
\author{Area 51: Shu-Ping Chien, Brock Smedley, W Keith Stirby Jr \\ 02 December 2017 \\ CS461, Senior Software Engineering Project, Fall 2017}


\begin{document}
\begin{titlepage}
\maketitle

\begin{abstract}


\thispagestyle{empty}
\end{abstract}
\end{titlepage}

\newpage
\tableofcontents

\newpage

\section{Introduction}

\subsection{Purpose}

Our project is required to provide a video output at near real-time from a 
multi-camera input utilizing an NVIDIA Jetson TX1 or TX2 system-on-module (SoM). 
The software produced will perform image processing and edge computing on the 
camera input to display a visually enhanced and stitched video output. 
Size, weight, power, and cost (SWaP-C) requirements for the project are due 
to its application being for UAS and VR/AR, and from our system utilizing a 
mass-produced SoM, particularly the Jetson TX1 or TX2.  

\subsection{Scope}
 
Software scope/requirements here. (Take pixel streams, do the hokey pokey, and 
we have low latency, wicked fast video at near real-time speed.)
 
Application specific hardware and software are vital for the project due the the 
system requirements and constraints of the project.
The NVIDIA Jetson TX1 and TX2 run on an Ubuntu based Linux for Tegra (L4T) operating 
system created specifically for producing customized imaging, installed on a 
GPU-accelerated dual-core CPU with dual Image Signal Processors (ISPs). 
NVIDIA provides the Jetson Software Development Pack (JetPack) 3.1 which has L4T 28.1 
and is capable of supporting a multitude of multimedia and image processing API. 
The cameras will utilize the most widely used camera interface for mobile applications, 
MIPI CSI-2, which is capable of supporting 1080p, 4k, and 8k video. 
A carrier board will connect the cameras to the Jetson TX1 or TX2 that also provides 
additional application-specific interfaces and peripherals. 


\subsection{Overview}

\section{Week 1}

\subsection{Summary}


\subsection{Retrospective}

\begin{tabular}{|p{0.3\linewidth}|p{0.3\linewidth}|p{0.3\linewidth}|}
   \hline
   \textbf{Positives} & \textbf{Deltas} & \textbf{Actions}\\ 
   \hline
   anything good that happened 
   & changes that need to be implemented 
   & specific actions that will be implemented in order to create the necessary 
   changes \\
   \hline
\end{tabular}

\section{Week 2}

\subsection{Summary}


\subsection{Retrospective}

\begin{tabular}{|p{0.3\linewidth}|p{0.3\linewidth}|p{0.3\linewidth}|}
   \hline
   \textbf{Positives} & \textbf{Deltas} & \textbf{Actions}\\ 
   \hline
   anything good that happened & changes that need to be implemented & specific actions that will be implemented in order to create the necessary changes \\
   \hline
\end{tabular}


\section{Week 3}

\subsection{Summary}


\subsection{Retrospective}

\begin{tabular}{|p{0.3\linewidth}|p{0.3\linewidth}|p{0.3\linewidth}|}
   \hline
   \textbf{Positives} & \textbf{Deltas} & \textbf{Actions}\\ 
   \hline
   anything good that happened & changes that need to be implemented & specific actions that will be implemented in order to create the necessary changes \\
   \hline
\end{tabular}

\subsection{Summary}

\section{Week 4}

\subsection{Summary}


\subsection{Retrospective}

\begin{tabular}{|p{0.3\linewidth}|p{0.3\linewidth}|p{0.3\linewidth}|}
   \hline
   \textbf{Positives} & \textbf{Deltas} & \textbf{Actions}\\ 
   \hline
   anything good that happened & changes that need to be implemented & specific actions that will be implemented in order to create the necessary changes \\
   \hline
\end{tabular}

\section{Week 5}

\subsection{Summary}


\subsection{Retrospective}


\begin{tabular}{|p{0.3\linewidth}|p{0.3\linewidth}|p{0.3\linewidth}|}
   \hline
   \textbf{Positives} & \textbf{Deltas} & \textbf{Actions}\\ 
   \hline
   anything good that happened & changes that need to be implemented & specific actions that will be implemented in order to create the necessary changes \\
   \hline
\end{tabular}

\section{Week 6}

\subsection{Summary}


\subsection{Retrospective}


\begin{tabular}{|p{0.3\linewidth}|p{0.3\linewidth}|p{0.3\linewidth}|}
   \hline
   \textbf{Positives} & \textbf{Deltas} & \textbf{Actions}\\ 
   \hline
   anything good that happened & changes that need to be implemented & specific actions that will be implemented in order to create the necessary changes \\
   \hline
\end{tabular}

\section{Week 7}

\subsection{Summary}


\subsection{Retrospective}


\begin{tabular}{|p{0.3\linewidth}|p{0.3\linewidth}|p{0.3\linewidth}|}
   \hline
   \textbf{Positives} & \textbf{Deltas} & \textbf{Actions}\\ 
   \hline
   anything good that happened & changes that need to be implemented & specific actions that will be implemented in order to create the necessary changes \\
   \hline
\end{tabular}

\section{Week 8}

\subsection{Summary}


\subsection{Retrospective}


\begin{tabular}{|p{0.3\linewidth}|p{0.3\linewidth}|p{0.3\linewidth}|}
   \hline
   \textbf{Positives} & \textbf{Deltas} & \textbf{Actions}\\ 
   \hline
   anything good that happened & changes that need to be implemented & specific actions that will be implemented in order to create the necessary changes \\
   \hline
\end{tabular}

\section{Week 9}

\subsection{Summary}


\subsection{Retrospective}


\begin{tabular}{|p{0.3\linewidth}|p{0.3\linewidth}|p{0.3\linewidth}|}
   \hline
   \textbf{Positives} & \textbf{Deltas} & \textbf{Actions}\\ 
   \hline
   anything good that happened & changes that need to be implemented & specific actions that will be implemented in order to create the necessary changes \\
   \hline
\end{tabular}

\section{Week 10}

\subsection{Summary}


\subsection{Retrospective}


\begin{tabular}{|p{0.3\linewidth}|p{0.3\linewidth}|p{0.3\linewidth}|}
   \hline
   \textbf{Positives} & \textbf{Deltas} & \textbf{Actions}\\ 
   \hline
   anything good that happened & changes that need to be implemented & specific actions that will be implemented in order to create the necessary changes \\
   \hline
\end{tabular}





\nocite{*}

\end{document}