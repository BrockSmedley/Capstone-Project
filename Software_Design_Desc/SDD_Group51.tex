\documentclass[letterpaper,10pt,serif,draftclsnofoot,onecolumn,compsoc,titlepage]{IEEEtran}

\usepackage{graphicx}                                        
\usepackage{amssymb}                                         
\usepackage{amsmath}                                         
\usepackage{amsthm}                                          
\usepackage{cite}
\usepackage{alltt}                                           
\usepackage{float}
\usepackage{color}
\usepackage[hyphens]{url}
\usepackage{pgfgantt}
\usepackage{rotating}
\usepackage{enumitem}
\usepackage{gensymb}
\usepackage[T1]{fontenc}

\usepackage{balance}
\usepackage[TABBOTCAP, tight]{subfigure}
\usepackage{enumitem}

\usepackage{geometry}
\geometry{margin=.75in}
\usepackage{hyperref}
\usepackage{breakurl}
%\usetikzlibrary{shapes, positioning, calc}
\usepackage{caption}
\usepackage{listings}
%\usepackage[utf8]{inputenc}
%pull in the necessary preamble matter for pygments output

%% The following metadata will show up in the PDF properties
\hypersetup{
   colorlinks = true,
   citecolor = black,
   linkcolor = black,
   urlcolor = black,
   breaklinks = true,
   pdfauthor = {Shu-Ping Chien, Brock Smedley, W Keith Striby Jr},
   pdfkeywords = {CS461 "Senior Project" Software Design Description},
   pdftitle = {CS461 Software Design Description},
   pdfsubject = {CS461 Software Design Description},
   pdfpagemode = UseNone
}

\parindent = 0.0 in
\parskip = 0.1 in
\title{Software Design Description: Multi-Camera, SoM Based, Real-Time Video Processing for UAS and VR/AR Applications}
\author{Area 51: Shu-Ping Chien, Brock Smedley, W Keith Stirby Jr \\ 01 December 2017 \\ CS461, Senior Software Engineering Project, Fall 2017}
\begin{document}
\begin{titlepage}
\maketitle
\begin{abstract}

abstract words \\

\thispagestyle{empty}
\end{abstract}
\end{titlepage}

%\newpage
%\tableofcontents

\newpage

\section{Frontispiece}

\subsection{Date of issue and status}

01 December 2017, In-progress \\

\subsection{Issuing Organization}

Rockwell Collins, CS Capstone Group 51, Oregon State University \\

\subsection{Authorship}

Shu-Ping Chien, Brock Smedley, W Keith Stirby Jr \\

\section{Introduction}

\subsection{Purpose}

This document is a general description of design concepts that will be used for our 
multi-camera, SoM based, real-time video processing for UAS and VR/AR applications. 
It is a reference to guide the development of our product.  \\

\subsection{Scope}

Our product will receive input from multiple cameras and then provide a video output at 
near real-time. Our software will receive the pixel data from the cameras and format 
the pixel streams so that image processing can occur. Image processing will then stitch 
the multiple streams of pixels being received to create a combined output. The software 
will be flashed onto the NVIDIA TX1/2, which receives input from the carrier 
board that is connected to the cameras. \\

\subsection{Overview}

The development and design of our product requires: hardware interface and system 
architecture, receiving camera input and formatting for image processing, and image 
processing for video output. The structure of our document reflects these three areas 
of our software required for our product. \\ 

\section{References}

\bibliographystyle{ieeetr}
\bibliography{SDD_Group51}

\section{System Overview}  

\subsection{Identified Stakeholders and Design Concerns}

Rockwell Collins is the primary customer of this product. The company provides 
engineering products in the aviation industry for commercial and military customers. 
Rockwell Collins is the primary user of the product and it is contributing 
to the development of a system that will be used in UAS and VR/AR applications. \\

\subsection{Hardware Context}

The hardware used in our product will be the NVIDIA Jetson TX1/2 as our SoM, 
a carrier board, and cameras. The input from cameras will be transferred through 
carrier boards, and the TX1/2 will format and process the input to produce a 
video output. The software we're developing will be on the TX1/2. \\

\subsection{Software Context}

The software pieces in this project include: GStreamer for transforming video input 
from CSI for image processing, and OpenGL development environment for image processing 
to produce the video output. The pipeline feature in GStreamer will reduce 
cost on time and storage to help achieve near real-time image processing. The OpenGL 
will stitch input images from GStreamer and print the output to the display device. \\

\section{System Architecture}

\subsection{Topic to design}

\subsubsection{subsection topic of topic to design}

words \\

\subsubsection{another subsection topic of topic to design}

words \\

\subsection{Making Camera Input Ready for Image Processing}

\subsubsection{Multimedia API}

The raw pixel data from the cameras will be sent through the CSI and must be converted 
to the BRG color space in the VI (Video Input) unit before being sent to the ISPs 
(Image Signal Processors). Application development can use either libargus API 
included in L4T or GStreamer plugin to prepare the raw pixel data in the VI unit, and 
therefore convert the data to a format that’s recognizable by the ISP. NVIDIA does not 
support V4L2 when using CSI cameras, but GStreamer does and therefore will be used in 
our software.  \\

\subsubsection{GStreamer}

GStreamer architecture utilizes pipelines to process media and connect the processing 
elements. An additional architecture GStreamer uses is plugin, which provides each 
processing element. \\

\paragraph{Installing GStreamer}

To install GStreamer 1.0 and plugins the following commands will be entered in the 
command line: \\

sudo add-apt-repository universe \\
sudo add-apt-repository multiverse \\
sudo apt-get update \\
sudo apt-get install gstreamer1.0-tools gstreamer1.0-alsa gstreamer1.0-plugins-base 
gstreamer1.0-plugins-good gstreamer1.0-plugins-bad gstreamer1.0-plugins-ugly 
gstreamer1.0-libav \\
sudo apt-get install libgstreamer1.0-dev libgstreamer-plugins-base1.0-dev 
libgstreamer-plugins-good1.0-dev libgstreamer-plugins-bad1.0-dev \\

\paragraph{GStreamer Plugins}

To display what a camera is capturing \texttt{nvcamerasrc} included in the piping, 
and is a plugin that allows options for our software to control ISP properties. \\

For converting video frames from the CSI camera the \texttt{videoconvert} plugin will 
be included in piping, along with \texttt{video/x-raw, format=(string){}} to specify 
details on the conversion input and output. \\


\nocite{*}
%\newpage
%\bibliographystyle{ieeetr}
%\bibliography{SDD_Group51}
\end{document}