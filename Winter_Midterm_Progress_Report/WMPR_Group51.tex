\documentclass[letterpaper,10pt,serif,draftclsnofoot,onecolumn,compsoc,titlepage]{IEEEtran}

\usepackage{graphicx}                                        
\usepackage{amssymb}                                         
\usepackage{amsmath}                                         
\usepackage{amsthm}                                          
\usepackage{cite}
\usepackage{alltt}                                           
\usepackage{float}
\usepackage{color}
\usepackage[hyphens]{url}
\usepackage{pgfgantt}
\usepackage{rotating}
\usepackage{enumitem}
\usepackage{gensymb}
\usepackage[T1]{fontenc}

\usepackage{balance}
\usepackage[TABBOTCAP, tight]{subfigure}
\usepackage{enumitem}

\usepackage{geometry}
\geometry{margin=.75in}
\usepackage{hyperref}
\usepackage{breakurl}
%\usetikzlibrary{shapes, positioning, calc}
\usepackage{caption}
\usepackage{listings}
%\usepackage[utf8]{inputenc}
%pull in the necessary preamble matter for pygments output

%% The following metadata will show up in the PDF properties
\hypersetup{
   colorlinks = true,
   citecolor = black,
   linkcolor = black,
   urlcolor = black,
   breaklinks = true,
   pdfauthor = {Shu-Ping Chien, Brock Smedley, W Keith Striby Jr},
   pdfkeywords = {CS461 "Senior Project" Progress Report},
   pdftitle = {CS461 Winter Midterm Progress Report},
   pdfsubject = {CS461 Winter Midterm Progress Report},
   pdfpagemode = UseNone
}

\parindent = 0.0 in
\parskip = 0.1 in
\title{Winter Midterm Progress Report: Multi-Camera, SoM Based, Real-Time Video Processing for UAS and VR/AR Applications}
\author{Area 51: Shu-Ping Chien, Brock Smedley, W Keith Stirby Jr \\ 08 February 2018 \\ CS461, Senior Software Engineering Project, Winter 2018}


\begin{document}
\begin{titlepage}
\maketitle

\begin{abstract}

This document highlights the group's progress made during the Winter 2018 term. \\
Update the rest \\
The 
project is introduced along with an explanation of the hardware and software 
surrounding our product's planned solution. Each week of the Fall 2017 term is 
summarized along with a retrospective. \\

\thispagestyle{empty}
\end{abstract}
\end{titlepage}

\newpage
\tableofcontents

\newpage

\section{Introduction}

\subsection{Purpose}
Our project is required to provide a video output at near real-time from a 
multi-camera input utilizing an NVIDIA Jetson TX1 or TX2 system-on-module (SoM). 
The software produced will perform image processing and edge computing on the 
camera input to display a visually enhanced and stitched video output. 
Size, weight, power, and cost (SWaP-C) requirements for the project are due 
to its application being for UAS and VR/AR, and from our system utilizing a 
mass-produced SoM, particularly the Jetson TX1 or TX2. \\

\subsection{Scope}
Application specific hardware and software are vital for the project due the the 
system requirements and constraints of the project.
The NVIDIA Jetson TX1 and TX2 run on an Ubuntu based Linux for Tegra (L4T) operating 
system created specifically for producing customized imaging, installed on a 
GPU-accelerated dual-core CPU with dual Image Signal Processors (ISPs). \\

L4T, being an Ubuntu variant, provides a friendly development interface with access to a 
large repository of additional software.
NVIDIA provides the Jetson Software Development Pack (JetPack) 3.1 which has L4T 28.1 
and is capable of supporting a multitude of multimedia and image processing API. Jetpack 
is installed on an Ubuntu host and then used to flash the TX2's memory with L4T and 
selected libraries. \\

The cameras will utilize the most widely used camera interface for mobile applications, 
MIPI CSI-2, which is capable of supporting 1080p, 4k, and 8k video. 
A carrier board will connect the cameras to the Jetson TX1 or TX2 that also provides 
additional application-specific interfaces and peripherals. \\

The software materials for this project will support the NVIDIA Jetson TX1 and TX2 and 
be able to produce fused images from cameras in near real-time. The media inputs from 
cameras through CSI will be captured in pipelines for distribution and transformation 
then sent to the image processor, and the process avoids to reload our input datas to 
reduce latency. The image processing software will utilize the images from media stream 
and produce stitched images as output to the display device.  \\

\subsection{Overview}
This document provides a recap of the progress made on our project during the 
Winter 2018 term. \\
Update this section \\
In addition to completing required documents and assignments the group primarily 
researched on hardware and software solutions necessary to complete the project 
successfully. Once hardware research was completed and procurement started the group 
began researching more into the design and development of software. The following 
sections reflect this progress with detailed weekly summaries and retrospectives. \\


\section{Shu-Ping}
Describe where you are currently on the project \\

Describe what you have left to do \\

Describe any problems that have impeded your progress, with any solutions \\

Include particularly interesting pieces of code \\

Include images of your project - screen shots, photos, etc \\

\section{Brock}
Describe where you are currently on the project \\

Describe what you have left to do \\

Describe any problems that have impeded your progress, with any solutions \\

Include particularly interesting pieces of code \\

Include images of your project - screen shots, photos, etc \\


\section{Keith}

\subsection{Current Project Status}
%Describe where you are currently on the project \\

As of the end of week five, the project had another unsuccessful attempt to demonstrate 
multi-camera input with the Jetson TX2 module. Previous attempts were with the TX2 module
connected to the development kit and the Auvidea J20 module which has six CSI-2 camera 
ports. Our most recent attempt was with the TX2 module connected to the Auvidea J106 
carrier board and the Auvidea M110 motherboard. Both of these attempts were receiving 
input from two or three Raspberry Pi v2.1 (IMX-219) cameras, and various GStreamer 
pipelines were input to produce a video output. \\

The group is waiting on an IMX274CS-X from Leopard Imaging, which has an 
adapter board allowing it to connect to the TX2 module. The adapter board can support 
up to six MIPI cameras simultaneously and also comes with the setup. The IMX274CS-X 
promises to be compatible with the Connect Tech Inc Spacely Carrier
Board and the Colorado Engineering Inc X-Carrier carrier board. This is helpful due to 
our struggles to obtain drivers and patches to support other TX2 module and carrier 
board configurations. \\


\subsection{Work Remaining for Project}
%Describe what you have left to do \\

Following the teams two attempts to produce multi-camera inputs with the TX2 module 
we're having to re-think our approach to project completion. The kernel setup, patches, 
and drivers for the TX2, J106, M110 setup needs to be re-attempted since we only tested 
this a few times during week five. This may require the TX2 module to be re-flashed a 
few times to attempt multiple kernel configurations with various patches and drivers 
installed. My confidence with our GStreamer pipelines to produce camera output is 
also low due to the information from Ridge Run that requires the use of 
\texttt{v4l2src} in pipelines to produce video output with the TX2, J106, M110 setup. 
Before attempting to use \textt{v4l2src} in our pipeline we had only been successful 
using \texttt{nvcamerasrc} in a pipeline, which was to produce video output from a 
single camera. More research needs to be performed on GStreamer to ensure that we've 
made adequate pipelines for video output from a multiple camera input setup. This will 
also ensure that our pipelines are adequate to test other TX2 module setups with our 
other carrier boards.\\



\subsection{Problems Experienced and Solutions (if applicable)}
%Describe any problems that have impeded your progress, with any solutions \\


\subsection{Code Involved in the Project}
%Include particularly interesting pieces of code \\


\subsection{Images of our Project}
%Include images of your project - screen shots, photos, etc \\



\nocite{*}
\end{document}
